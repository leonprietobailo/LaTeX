\documentclass{report}
\usepackage[spanish]{babel}
\usepackage[letterpaper,top=2cm,bottom=2cm,left=3cm,right=3cm,marginparwidth=1.75cm]{geometry}

% Useful packages
\usepackage{amsmath}
\usepackage{graphicx}
\usepackage[colorlinks=true, allcolors=blue]{hyperref}
\usepackage{fancyhdr}

\pagestyle{fancy}
\fancyhead[L]{Sistemas de propulsión, actuaciones e integración.}
\fancyhead[R]{León E. Prieto Bailo}

\setcounter{secnumdepth}{0}
\setlength{\parindent}{0pt}

\title{Sistemas de propulsión, actuaciones e integración.}
\author{León E. Prieto Bailo}

\begin{document}
\maketitle

% \begin{abstract}
% Your abstract.
% \end{abstract}

\section{Enunciado}

Una turbina de gas industrial opera a nivel del mar. Absorbe 66.67 Kg/s de aire. El compresor opera con una relación de compresión de 18:1 y una eficiencia del 99\%. La temperatura total de salida de cámara de combustión es 1456 K, y la eficiencia de la cámara de combustión es del 96\%, perdiendo un 4\% de presión total, y la turbina tiene un rendimiento del 91\%. La entrada de aire pierde un 4\% de presión total. L = 42.8 MJ/kg\\

Calcular La potencia obtenida, el consumo específico, y la eficiencia térmica.\\

Se decide estudiar el compresor, y se observa que no tiene álabes guía, que el mach axial de entrada es 0.5, y que el ángulo (relativo) de entrada en los álabes del rotor para dos casos $\beta_1$ = 46 ºC, y $\beta_2$ = 51 ºC. Y se sabe también que el diseño incluye h/c=1.5. Suponiendo para ambos casos que se diseña con $W_2/W_1$ = 0.7, la velocidad circunferencial de giro en la punta del álabe es 450 m/s, la viscosidad cinemática es $1.73\cdot10^{-5}$ $m^2$/s, el factor de difusión en el rotor es de valor 0.45 (línea media), y que la salida del aire de cada estátor es 0 ºC y la velocidad axial es constante\\

Calcular y comparar:\

\begin{itemize}
    \item Ángulos relativos y absolutos de la corriente de aire, deflexión de la corriente en rotor y estátor (en línea media)
    \item Número de escalones de todo el compresor
    \item Relación de compresión del escalón
    \item Área frontal de la entrada (Radio exterior e interior del rotor)
    \item Velocidad de giro del eje
    \item Cuerda del álabe, y número de álabes del rotor
    \item Número de Reynolds en el álabe basado en la cuerda
    \item Grado de reacción
\end{itemize}
\newpage
\section{Resolución}
\subsection{Primera parte}
\subsubsection{Calcular la potencia obtenida, el consumo específico, y la eficiencia térmica.}
Asumiendo condiciones de atmosfera estandar:
\begin{gather}
  T_0 = 288\:K\\
  \vspace{1cm}
  P_0 = 101,325\:KPa
\end{gather}
Asumiendo que la turbina de gas industrial está estatica:
\begin{gather}
  T_{t0} = T_0\\
  P_{t0} = P_0
\end{gather}
Considerando las perdidas a la entrada de la turbomaquina:
\begin{gather}
  P_{t2} = P_{t0} \pi_i=97,272\:KPa\\
  T_{t2} = T_{t0} \pi_i^{\frac{\gamma-1}{\gamma}} = 279.72\:K
\end{gather}
En la etapa del compresor:
\begin{gather}
  P_{t3} = P_{t2} \pi_c=1,750\:MPa\\
  \eta = \frac{T_{t3}'-T_{t2}}{T_{t3}-T_{t2}}=\frac{\pi_c^\frac{\gamma-1}{\gamma}-1}{\frac{T_{t3}}{T_{t2}}-1}\xrightarrow{}T_{t3} = 642,44\:K
\end{gather}
Conociendo el salto de temperaturas en el compresor, podemos hallar la potencia obtenida:
\begin{gather}
  P=\Dot{m_a}C_p(T_{t3}-T_{t2})=\boxed{24,302\:MW}
\end{gather}
Para la salida de la camara de combustión:
\begin{gather}
  T_{t4} = 1456\:K\\
  P_{t4} = P_{t3} \pi_{cc}=1,680\:MPa
\end{gather}
Planteando el balance energetico y sabiendo que el compresor es alimentado por el movimento de la turbina: 
\begin{gather}
  \Dot{m_a}C_p(T_{t3}-T_{t2})=\Dot{m_t}C_p(T_{t4}-T_{t5})
\end{gather}
Teniendo en cuenta la eficiencia adiabatica de la camara de combustión:
\begin{gather}
  \eta_{cc}=\frac{T_{t4}-T_{t3}}{T_{t4}'-T_{t3}}\xrightarrow{}T_{t4}'=T_{t3}+\frac{T_{t4}-T_{t3}}{\eta_{cc}}=1489.9\:K
\end{gather}
Para obtener el valor del flujo masico total, es necesario calcular el consumo de combustible. Para hacerlo, planteamos el incremento de entalpias en la camara de combustión:
\begin{gather}
  h_{04}=h_{03}+fL
\end{gather}
Lo cual, se puede reescribir como:
\begin{gather}
  f=\frac{\frac{T_{t4}'}{T_{t3}}-1}{\frac{L}{C_p T_{t3}}-\frac{T_{t4}'}{T_{t3}}}=0,0206
\end{gather}
Teniendo el valor de $f$, podemos hallar la temperatura a la salida de la turbina $(T_{t5})$ como:
\begin{gather}
  T_{t5}=T_{t4}-\frac{1}{1+f}(T_{t3}-T_{t2})=1100.61\:K
\end{gather}
Mientras que, el valor de la presion total a la salida de la turbina $(P_{t5})$ podemos hallarla con la expresion del rendimiento adiabático $(\eta_T)$:
\begin{gather}
  \eta_T = \frac{T_{t4}-T_{t5}}{T_{t4}-T_{t5}'}=\frac{1-\frac{T_{t5}}{T_{t4}}}{1-\left(\frac{P_t5}{P_t4}\right)^\frac{\gamma-1}{\gamma}}\xrightarrow{}P_{t5}=P_{t4}\left(1-\frac{1-\frac{T_{t5}}{T_{t4}}}{\eta_T}\right)^\frac{\gamma}{\gamma-1}=1,681\:MPa
\end{gather}
Asumiento comportamiento isentalpico en la salida:
\begin{gather}
  T_{t7} = T_{t5}\\
  P_{t7} = P_{t5}
\end{gather}
Estudiando las condiciones criticas de la tobera:
\begin{gather}
  \frac{P_{t7}}{P^*}=\left(\frac{\gamma+1}{2}\right)^\frac{\gamma}{\gamma-1}=1,893\\
  \frac{P_{t7}}{P_0}=16,42 >> 1,893
\end{gather}
Por lo tanto nos encontramos en condiciones críticas, por lo que:
\begin{gather}
  M_7 = 1\\
  T_7 = T_{t7} \frac{1}{1 + \frac{\gamma - 1}{2}} = 917,17\:K\\
  P_7 = P_{t7} \frac{1}{1 + \left(\frac{\gamma - 1}{2}\right)^\frac{\gamma}{\gamma-1}}=87,909\:MPa
\end{gather}
Calculando la velocidad de salida $(V_7)$, la densidad $(\rho_7)$ y el area de la tobera $(A_7)$:
\begin{gather}
  V_7 = \sqrt{\gamma R T_7}=607,06\:m/s\\
  \rho_7 = \frac{P_7}{R T_7}=3,34\:kg/m^3\\
  A_7 = \frac{\Dot{m} (1+f)}{\rho_7 V_7}=0,0336\:m^2
\end{gather}
Calculando el empuje y el consumo espeficico:
\begin{gather}
  T = \Dot{m} (1+f) V_7 + (P_7-P_0)A_7=67,441\:kN\\
  TSFC = \frac{\Dot{m} (1+f)}{T}=\boxed{20.384\:\ g/kN/s}
\end{gather}
Podemos obtener la eficiencia termica de la turbina de gas relacionando la potencia aprovechada por la turbina con la potencia generada por la quema de combustible en la camara de combustion.
\begin{gather}
  \eta_T = \frac{P}{\Dot{m}_f L}=\boxed{0.413}
\end{gather}
\newpage
\subsection{Segunda parte}
\subsubsection{Ángulos relativos y absolutos de la corriente de aire, deflexión de la corriente en rotor y estátor (en línea media)}
\subsubsection{Número de escalones de todo el compresor}
\subsubsection{Relación de compresión del escalón}
\subsubsection{Área frontal de la entrada (Radio exterior e interior del rotor)}
\subsubsection{Velocidad de giro del eje}
\subsubsection{Cuerda del álabe, y número de álabes del rotor}
\subsubsection{Número de Reynolds en el álabe basado en la cuerda}
\subsubsection{Grado de reacción}
\end{document}